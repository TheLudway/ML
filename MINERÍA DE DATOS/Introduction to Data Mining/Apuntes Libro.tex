\documentclass{article}

\usepackage{amsmath, amsthm, amssymb, amsfonts}
\usepackage{thmtools}
\usepackage{graphicx}
\usepackage{setspace}
\usepackage{geometry}
\usepackage{float}
\usepackage[hidelinks]{hyperref}
\usepackage[utf8]{inputenc}
\usepackage[spanish]{babel}
\usepackage{framed}
\usepackage[dvipsnames]{xcolor}
\usepackage{tcolorbox}
\usepackage{tikz}
\usepackage{caption}
\usepackage{longtable}
\usepackage{pdflscape}
\usepackage{svg}
\usepackage{subcaption}
\usepackage{caption}
\usepackage{multirow}
\usepackage{array}
\usepackage{listings}

\colorlet{LightGray}{White!90!Periwinkle}
\colorlet{LightOrange}{Orange!15}
\colorlet{LightGreen}{Green!15}



\newcommand{\HRule}[1]{\rule{\linewidth}{#1}}

\declaretheoremstyle[name=Theorem,]{thmsty}
\declaretheorem[style=thmsty,numberwithin=section]{theorem}
\tcolorboxenvironment{theorem}{colback=LightGray}

\declaretheoremstyle[name=Proposition,]{prosty}
\declaretheorem[style=prosty,numberlike=theorem]{proposition}
\tcolorboxenvironment{proposition}{colback=LightOrange}

\declaretheoremstyle[name=Principle,]{prcpsty}
\declaretheorem[style=prcpsty,numberlike=theorem]{principle}
\tcolorboxenvironment{principle}{colback=LightGreen}

\newcolumntype{L}[1]{>{\raggedleft\let\newline\\\arraybackslash\hspace{0pt}}m{#1}}
\newcolumntype{C}[1]{>{\centering\let\newline\\\arraybackslash\hspace{0pt}}m{#1}}
\newcolumntype{R}[1]{>{\raggedright\let\newline\\\arraybackslash\hspace{0pt}}m{#1}}

\setstretch{1.2}
\geometry{
    textheight=9in,
    textwidth=5.5in,
    top=1in,
    headheight=12pt,
    headsep=25pt,
    footskip=30pt
}

\lstdefinestyle{bashstyle}{
    language=bash,
    basicstyle=\ttfamily,
    backgroundcolor=\color{gray!10},
    keywordstyle=\color{blue},
    commentstyle=\color{green!40!black},
    stringstyle=\color{red},
    showstringspaces=false,
    numbers=left,
    numberstyle=\tiny\color{gray},
    breaklines=true,
    breakatwhitespace=true,
    frame=tb,
    rulecolor=\color{black!70},
    framerule=0.5pt,
    tabsize=4,
    captionpos=b
}

\lstdefinestyle{javastyle}{
    language=Java,
    basicstyle=\ttfamily,
    backgroundcolor=\color{gray!10},
    keywordstyle=\color{blue},
    commentstyle=\color{green!40!black},
    stringstyle=\color{red},
    showstringspaces=false,
    numbers=left,
    numberstyle=\tiny\color{gray},
    breaklines=true,
    breakatwhitespace=true,
    frame=tb,
    rulecolor=\color{black!70},
    framerule=0.5pt,
    tabsize=4,
    captionpos=b
}

% ------------------------------------------------------------------------------

\begin{document}

\title{ \normalsize \textsc{}
	\\ [2.0cm]
	\HRule{1.5pt} \\
	\LARGE \textbf{\uppercase{Introduction To Data Mining}
		\HRule{2.0pt} \\ [0.6cm] \LARGE{Second Edition} \vspace*{10\baselineskip}}
}
\date{}
\author{\textbf{Alvarado Becerra Ludwig} \\
	El ingeniero más lamentable}

\maketitle
\thispagestyle{empty}
\newpage

\tableofcontents
\thispagestyle{empty}
\newpage
\setcounter{page}{1}

\section{Introduction}

Se habla de que vivimos en un mundo en el cual la cantidad de datos que se están sacando son muy grandes, y se están sacando de muchos campos diferentes, se habla de cómo se puede asociar el \textit{IoT (Internet Of Things)} para recolectar datos, cómo en la medicina, ingeniería, biología, medio ambiente, entre otros campos, se usan diferentes técnicas de minería de datos.

\subsection{¿Qué es la Minería de Datos?}
Es el proceso de descubrir información valiosa en grandes repositorio de datos. Sus técnicas son desplegadas sobre grandes \textit{datasets} para encontrar información que de otra forma hubiera permanecido desconocida. También, dan la capacidad de predecir la salida de una futura observación.

La minería de datos forma parte de algo más general conocido como \textbf{Knowledge Discovery in Databases (KDD)}, que se basa en transformar datos en información importante. Este proceso se puede dividir en estos pasos:

\begin{enumerate}
  \item Ingreso de datos.
  \item Procesamiento de datos.
        \begin{itemize}
          \item Selección de características.
          \item Reducción de dimensionalidad.
          \item Normalización.
          \item Separación de datos
        \end{itemize}
  \item Minería de datos.
  \item Posprocesamiento.
        \begin{itemize}
          \item Filtros de patrones.
          \item Visualización.
          \item Interpretación de patrones.
        \end{itemize}
  \item Información
\end{enumerate}

El preprocesamiento de los datos es una de las tareas más duras y que puede llevar más tiempo debido a que toca utilizar diferentes fuentes de datos, transformar los datos de entrada en un formato apropiado para el análisis, limpiar datos y remover ruido y observaciones duplicadas.

Por otra parte, el posprocesamiento, se utiliza para transmitir la información que se sacó de la minería de datos; de mis resultados ¿Qué le da un valor agregado al negocio? ¿Qué decisión se puede tomar con esto?. Sus técnicas son la visualización para poder transmitir esto a otros campos de la empresa.


\subsection{Retos motivacionales}
\begin{itemize}
  \item \textbf{Escalabilidad:} Debido a que hay \textit{datasets} de gigabytes, petabytes e incluso más grandes, los algoritmos de minería deben ser capaces de abordar esas dificultades reduciendo el tiempo de computo.
  \item \textbf{Alta dimensionalidad:} A menudo cuando se recolectan más datos, aparecen más variables. Las técnicas tradicionales de análisis de datos funcionan para \textit{datasets} con pocas variables, pero en la era moderna estos están siendo cada vez más grandes y se necesitan algoritmos que los puedan manejar eficientemente.
  \item \textbf{Datos heterogéneos y complejos:} Toca lidiar algunas veces con datos que son muy parecidos entre sí, a su vez, a medida que crece el mundo digital, crecen los datos complejos como; texto, imagenes, vídeos, audio, ADN con estructuras en 3D, datos espaciales, datos del clima. Algunas técnicas son de apreciar para realizar diferentes relaciones entre los datos.
  \item \textbf{Propiedad de los datos y distribución:} Los datos vienen de diferentes recursos, entidades y organizaciones. Por eso mismo, toca saber utilizar y organizar los datos cuando se están juntando de varias fuentes. A su vez, toca tener en cuenta la licencia de los datos y la privacidad.
  \item \textbf{Análisis no tradicional:} El tradicional viene de la estadística, donde se propone una hipótesis y un experimento es diseñado para recolectar los datos, y luego estos se analizan con respecto a la hipótesis. Este proceso es costoso en tiempo y la situación actual requiere analizar miles de hipótesis.
\end{itemize}   

\subsection{Tareas de Minería de Datos}

\begin{itemize}
  \item \textbf{Tareas predictivas:} Se aplican con el objetivo de predecir el valor de un atributo en concreto basado en el valor de otros atributos. El atributo a predecir se conoce como \textbf{variable dependiente}, mientras que los atributos usados para la predicción se conocen como \textbf{variables independientes}.
  \item \textbf{Tareas descriptivas:} El objetivo es es sacar patrones (correlaciones, tendencias, clusters, trayectorias, anomalías) que hacen relaciones de entre los datos.
\end{itemize}

\end{document}
