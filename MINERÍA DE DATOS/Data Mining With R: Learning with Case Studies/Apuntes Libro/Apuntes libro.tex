\documentclass{article}

\usepackage{amsmath, amsthm, amssymb, amsfonts}
\usepackage{thmtools}
\usepackage{graphicx}
\usepackage{setspace}
\usepackage{geometry}
\usepackage{float}
\usepackage[hidelinks]{hyperref}
\usepackage[utf8]{inputenc}
\usepackage[spanish]{babel}
\usepackage{framed}
\usepackage[dvipsnames]{xcolor}
\usepackage{tcolorbox}
\usepackage{tikz}
\usepackage{caption}
\usepackage{longtable}
\usepackage{xurl}
\usepackage{pdflscape}
\usepackage{listings}
\usepackage{array}
\usepackage{multirow}





\colorlet{LightGray}{White!90!Periwinkle}
\colorlet{LightOrange}{Orange!15}
\colorlet{LightGreen}{Green!15}


\newcommand{\HRule}[1]{\rule{\linewidth}{#1}}

\declaretheoremstyle[name=Theorem,]{thmsty}
\declaretheorem[style=thmsty,numberwithin=section]{theorem}
\tcolorboxenvironment{theorem}{colback=LightGray}

\declaretheoremstyle[name=Proposition,]{prosty}
\declaretheorem[style=prosty,numberlike=theorem]{proposition}
\tcolorboxenvironment{proposition}{colback=LightOrange}

\declaretheoremstyle[name=Principle,]{prcpsty}
\declaretheorem[style=prcpsty,numberlike=theorem]{principle}
\tcolorboxenvironment{principle}{colback=LightGreen}

\setstretch{1.2}
\geometry{
    textheight=9in,
    textwidth=5.5in,
    top=1in,
    headheight=12pt,
    headsep=25pt,
    footskip=30pt
}

\definecolor{dkgreen}{rgb}{0,0.6,0}
\definecolor{gray}{rgb}{0.5,0.5,0.5}
\definecolor{mauve}{rgb}{0.58,0,0.82}
\lstset{language=SQL,
  basicstyle={\small\ttfamily},
  belowskip=3mm,
  breakatwhitespace=true,
  breaklines=true,
  classoffset=0,
  columns=flexible,
  commentstyle=\color{dkgreen},
  framexleftmargin=0.25em,
  frameshape={}{yy}{}{}, %To remove to vertical lines on left, set `frameshape={}{}{}{}`
  keywordstyle=\color{blue},
  numbers=none, %If you want line numbers, set `numbers=left`
  numberstyle=\tiny\color{gray},
  showstringspaces=false,
  stringstyle=\color{mauve},
  tabsize=3,
  xleftmargin =1em
}

\definecolor{codegreen}{rgb}{0,0.6,0}
\definecolor{codegray}{rgb}{0.5,0.5,0.5}
\definecolor{codepurple}{rgb}{0.58,0,0.82}
\definecolor{backcolour}{rgb}{0.95,0.95,0.92}

\lstdefinestyle{RStyle}{
    backgroundcolor=\color{backcolour},
    commentstyle=\color{codegreen},
    keywordstyle=\color{magenta},
    numberstyle=\tiny\color{codegray},
    stringstyle=\color{codepurple},
    basicstyle=\ttfamily\footnotesize,
    breakatwhitespace=false,
    breaklines=true,
    captionpos=b,
    keepspaces=true,
    numbers=left,
    numbersep=5pt,
    showspaces=false,
    showstringspaces=false,
    showtabs=false,
    tabsize=2
  }

  \lstdefinestyle{shellscript}{
  language=bash,
  basicstyle=\ttfamily\small,         % Basic font style
  keywordstyle=\color{blue}\bfseries, % Keywords style
  commentstyle=\color{gray},          % Comments style
  stringstyle=\color{red},            % Strings style
  backgroundcolor=\color{lightgray!20}, % Background color
  showstringspaces=false,             % Don't show spaces in strings
  numbers=left,                       % Line numbers on the left
  numberstyle=\tiny\color{gray},      % Line number style
  stepnumber=1,                       % Step between two line numbers
  frame=single,                       % Frame around the code
  frameround=tttt,                    % Rounded corners
  breaklines=true,                    % Automatic line breaking
  captionpos=b,                       % Position of the caption
  tabsize=2,                          % Tab size
  escapeinside={(*@}{@*)},            % Escape inside code
  morekeywords={sudo, cd, ls, mkdir, rm, rmdir, cp, mv, echo, cat, apt, install,
              get, nano} % More keywords
}


\newcolumntype{L}[1]{>{\raggedright\let\newline\\\arraybackslash\hspace{0pt}}m{#1}}
\newcolumntype{C}[1]{>{\centering\let\newline\\\arraybackslash\hspace{0pt}}m{#1}}
\newcolumntype{R}[1]{>{\raggedleft\let\newline\\\arraybackslash{0pt}}m{#1}}


\begin{document}


% ------------------------------------------------------------------------------
% Cover Page and ToC
% ------------------------------------------------------------------------------

\title{ \normalsize \textsc{}
		\\ [2.0cm]
		\HRule{1.5pt} \\
		\LARGE \textbf{\uppercase{Data Mining With R}
		\HRule{2.0pt} \\ [0.6cm] \LARGE{Learning with Case Studies} \vspace*{10\baselineskip}}
		}
\date{}
\author{\textbf{Alvarado Becerra Ludwig} \\
		Libro Luis Torgo\\
		\today}

\maketitle
\newpage

\tableofcontents
\newpage

\section{\textit{Introduction}}

En \texttt{R} existen varios paquetes, para este libro se va a utilizar la base de datos de \texttt{MySQL}, es decir, \texttt{RMySQL}. Tan sólo se necesita ejecutar el siguiente comando:
\begin{lstlisting}[language=R]
install.packages('RMySQL')
\end{lstlisting}

Para realizar la instalación correcta del páquete toca seguir el siguiente \textit{script} en \textit{shell}
\begin{lstlisting}[style=shellscript]
sudo apt-get update
sudo apt-get install libmariadb-dev
\end{lstlisting}

No olvidar que toca tener instalado \textit{MySQL} en el computador, lo cual se puede hacer fácilmente desde la
página web de la propia \textit{MySQL}.

Así, se puede ejecutar el \textit{script} de \texttt{R} anteriormente mencionado (se debe ejecutar en un entorno de \texttt{R}
que tenga privilegios de administrador). Se sabe que está bien instalado por la siguiente salida:

\begin{lstlisting}[language=R]
* DONE (RMySQL)

The downloaded source packages are in
	'/tmp/RtmpQ7XQQ5/downloaded_packages'
\end{lstlisting}

La función \lstinline[language=R]{install.packages()} tiene múltiples parámetros, lo que hace es buscar el repositorio con la
ventana más cercana a algún CRAN (\textit{Comprehensive R Archive Network}).

A continuación, se va a instalar el paquete asociado con el libro de \textit{Data Mining With R: Learning wit Case Studies}. Esto ayudará a tener funciones ya hechas y también algunos \textit{Datasets} que se utilizarán más adelante.
\begin{lstlisting}
install.package('DMwR')
\end{lstlisting}

El paquete no se encuentra disponible en CRAN. Sin embargo, se comenta que una posible alternativa al paquete es con el siguiente:
\begin{lstlisting}
install.package('grt')
\end{lstlisting}

También, se tiene el \textit{mirror} que había en el CRAN, que se presenta en el siguiente enlace
\url{https://github.com/cran/DMwR}. Por si en algún momento se necesita de algún archivo o función.

Algunos comandos para el manejo de paquetes son:
\begin{itemize}
  \item \lstinline[language=R]{installed.packages()}, se pueden ver los paquetes que están instalados actualmente en el computador
  \item \lstinline[language=R]{library()}, lo mismo que el anterior pero más amigable a la vista.
  \item \lstinline[language=R]{old.packages()}, verifica si hay versiones más nuevas en el CRAN.
  \item \lstinline[language=R]{update.packages()}, actualiza todos los paquetes que se tienen instalados.
\end{itemize}




\end{document}
