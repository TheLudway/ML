\documentclass{article}

\usepackage{amsmath, amsthm, amssymb, amsfonts}
\usepackage{thmtools}
\usepackage{graphicx}
\usepackage{setspace}
\usepackage{geometry}
\usepackage{float}
\usepackage[hidelinks]{hyperref}
\usepackage[utf8]{inputenc}
\usepackage[spanish]{babel}
\usepackage{framed}
\usepackage[dvipsnames]{xcolor}
\usepackage{tcolorbox}
\usepackage{tikz}
\usepackage{caption}
\usepackage{longtable}
\usepackage{xurl}
\usepackage{pdflscape}
\usepackage{listings}
\usepackage{array}
\usepackage{multirow}





\colorlet{LightGray}{White!90!Periwinkle}
\colorlet{LightOrange}{Orange!15}
\colorlet{LightGreen}{Green!15}


\newcommand{\HRule}[1]{\rule{\linewidth}{#1}}

\declaretheoremstyle[name=Theorem,]{thmsty}
\declaretheorem[style=thmsty,numberwithin=section]{theorem}
\tcolorboxenvironment{theorem}{colback=LightGray}

\declaretheoremstyle[name=Proposition,]{prosty}
\declaretheorem[style=prosty,numberlike=theorem]{proposition}
\tcolorboxenvironment{proposition}{colback=LightOrange}

\declaretheoremstyle[name=Principle,]{prcpsty}
\declaretheorem[style=prcpsty,numberlike=theorem]{principle}
\tcolorboxenvironment{principle}{colback=LightGreen}

\setstretch{1.2}
\geometry{
    textheight=9in,
    textwidth=5.5in,
    top=1in,
    headheight=12pt,
    headsep=25pt,
    footskip=30pt
}

\definecolor{dkgreen}{rgb}{0,0.6,0}
\definecolor{gray}{rgb}{0.5,0.5,0.5}
\definecolor{mauve}{rgb}{0.58,0,0.82}
\lstset{language=SQL,
  basicstyle={\small\ttfamily},
  belowskip=3mm,
  breakatwhitespace=true,
  breaklines=true,
  classoffset=0,
  columns=flexible,
  commentstyle=\color{dkgreen},
  framexleftmargin=0.25em,
  frameshape={}{yy}{}{}, %To remove to vertical lines on left, set `frameshape={}{}{}{}`
  keywordstyle=\color{blue},
  numbers=none, %If you want line numbers, set `numbers=left`
  numberstyle=\tiny\color{gray},
  showstringspaces=false,
  stringstyle=\color{mauve},
  tabsize=3,
  xleftmargin =1em
}


\newcolumntype{L}[1]{>{\raggedright\let\newline\\\arraybackslash\hspace{0pt}}m{#1}}
\newcolumntype{C}[1]{>{\centering\let\newline\\\arraybackslash\hspace{0pt}}m{#1}}
\newcolumntype{R}[1]{>{\raggedleft\let\newline\\\arraybackslash{0pt}}m{#1}}


\begin{document}


% ------------------------------------------------------------------------------
% Cover Page and ToC
% ------------------------------------------------------------------------------

\title{ \normalsize \textsc{}
		\\ [2.0cm]
		\HRule{1.5pt} \\
		\LARGE \textbf{\uppercase{Apuntes libro}
		\HRule{2.0pt} \\ [0.6cm] \LARGE{Introducción a la Minería de Datos} \vspace*{10\baselineskip}}
		}
\date{}
\author{\textbf{Alvarado Becerra Ludwig} \\ 
		Libro de José Hernández Orallo\\
		\today}

\maketitle
\newpage

\tableofcontents
\newpage

\section{¿Qué es la minería de datos?}
La minería de datos tiene su fundemantación en la estadística y en el \textit{Machine Learning}. Utiliza grandes cantidades de datos para generar conocimiento a partir de los datos, mientras que en otras ramas de estadística la idea era organizar los datos y obtener información a partir de ella a partir de consultas como lo era en el caso de SQL. En la minería de datos se buscan patrones que puedan ayudarnos a generar conocimiento a partir de esos datos, es ampliamente útil en todo tipo de industria, ingeniería, campo de la salud, todo lado!

Según el autor, el objetivo de la minería de datos es convertir datos en conocimiento. Para eso mismo, expone unos ejemplos, de los cuales voy a detallar sólo en uno que fue el que más me gustó ;)

\subsection{Determinar grupos diferenciados de empleados}

El departamento de recursos humanos de una gran empresa desea categorizar a sus empleados en distintos grupos con el objetivo de entender mejor su comportamiento y tratarlos de manera adecuada. Para ello dispone en sus bases de datos de información sobre los mismos (sueldo, estado civil, sii tiene coche, número de hijos, si su casa es propia o de alquier, si está sindicado, número de bajas al año, antigüedad y sexo). La tabla \ref{tab:datos-empleados} muestra algunos de los registros de su base de datos.

\begin{longtable}{|C{0.025\textwidth}|C{0.06\textwidth}|C{0.05\textwidth}|C{0.04\textwidth}|C{0.06\textwidth}|C{0.1\textwidth}|C{0.09\textwidth}|C{0.09\textwidth}|C{0.09\textwidth}|C{0.08\textwidth}|}
  \hline
  \textbf{id} & \textbf{Sueldo} & \textbf{Casado} &  \textbf{Coche} & \textbf{Hijos} & \textbf{Alq/prop} & \textbf{Sindicado} & \textbf{Bajas/Año} & \textbf{Antigüedad} & \textbf{Sexo} \\ \hline
  1 & 1.000 & Sí & No & 0 & Alquiler & No & 7 & 15 & H \\ \hline
  2 & 2.000 & No & Sí & 1 & Alquiler & Sí & 3 & 3 & M \\ \hline
  3 & 1.500 & Sí & Sí & 2 & Prop & Sí & 5 & 10 & H \\ \hline
  4 & 3.000 & Sí & Sí & 1 & Alquiler & No & 15 & 7 & M \\ \hline
  5 & 1.000 & Sí & Sí & 0 & Prop & Sí & 1 & 6 & H \\ \hline
  6 & 4.000 & No & Sí & 0 & Alquiler & Sí & 3 & 16 & M \\ \hline
  7 & 2.500 & No & No & 0 & Alquiler & Sí & 0 & 8 & H \\ \hline
  8 & 2.000 & No & Sí & 0 & Prop & Sí & 2 & 6 & M \\ \hline
  9 & 2.000 & Sí & Sí & 3 & Prop & No & 7 & 5 & H \\ \hline
  10 & 3.000 & Sí & Sí & 2 & Prop & No & 1 & 20 & H \\ \hline
  11 & 5.000 & No & No & 0 & Alquiler & No & 2 & 12 & M \\ \hline
  12 & 800 & Sí & Sí & 2 & Prop & No & 3 & 1 & H \\ \hline
  13 & 2.000 & No & No & 0 & Alquiler & No & 27 & 5 & M \\ \hline
  14 & 1.000 & No & Sí & 0 & Alquiler & Sí & 0 & 7 & H \\ \hline 
  15 & 800 & No & Sí & 0 & Alquiler & No & 3 & 2 & H \\ \hline
  $\vdots$ &  $\vdots$ &  $\vdots$ &  $\vdots$ &  $\vdots$ &  $\vdots$ &  $\vdots$ &  $\vdots$ &  $\vdots$ &  $\vdots$ \\ \hline
  \caption{Datos de los empleados}
  \label{tab:datos-empleados}
\end{longtable}

Un sistema de minería de datos podría obtener tres grupos con la siguiente descripción:

\begin{tcolorbox}
  {\LARGE \textbf{Grupo 1:}}
  \begin{itemize}
  \item Sueldo: 1.535,2 €.
  \item Casado: No -$>$ 0,777 | Sí -$>$ 0,223
  \item Coche: No -$>$ 0,82 | Sí -$>$ 0,18
  \item Hijos: 0,05
  \item Alq/Prop: Alquiler -$>$ 0,99 | Propia -$>$ 0,01
  \item Sindic.: No -$>$ 0,8 | Sí -$>$ 0,2
  \item Bajas/Año: 8,3
  \item Antigüedad: 8,7
  \item Sexo: H -$>$ 0,61 | M -$>$ 0,39
  \end{itemize}
\end{tcolorbox}

\begin{tcolorbox}
  {\LARGE \textbf{Grupo 2:}}
  \begin{itemize}
  \item Sueldo: 1.428,7 €.
  \item Casado: No -$>$ 0,98 | Sí -$>$ 0,02
  \item Coche: No -$>$ 0,01 | Sí -$>$ 0,99
  \item Hijos: 0,3
  \item Alq/Prop: Alquiler -$>$ 0,75 | Propia -$>$ 0,25
  \item Sindic.: Sí -$>$ 1,0
  \item Bajas/Año: 2,3
  \item Antigüedad: 8
  \item Sexo: H -$>$ 0,25 | M -$>$ 0,75
  \end{itemize}
\end{tcolorbox}

\begin{tcolorbox}
  {\LARGE \textbf{Grupo 3:}}
  \begin{itemize}
  \item Sueldo: 1.233,8 €.
  \item Casado: Sí -$>$ 1,0
  \item Coche: No -$>$ 0,05 | Sí -$>$ 0,95
  \item Hijos: 2,3
  \item Alq/Prop: Alquiler -$>$ 0,17 | Propia -$>$ 0,83
  \item Sindic.: No -$>$ 0,67 | Sí -$>$ 0,33
  \item Bajas/Año: 5,1
  \item Antigüedad: 8,1
  \item Sexo: H -$>$ 0,83 | M -$>$ 0,17
  \end{itemize}
\end{tcolorbox}

Estos grupos podrían ser interpretados por el departamento de recursos humanos de la siguiente manera:
\begin{itemize}
\item Grupo 1: sin hijos y con vivienda de alquiler. Poco sindicados. Muchas bajas.
\item Grupo 2: sin hijos y con coche. Muy sindicados. Pocas bajas. Normalmente son mujeres y viven en casas de alquiler.
\item Grupo 3: con hijos, casados y con coche. Mayoritariamente hombres propietarios de su vivienda. Poco sindicados.
\end{itemize}


\subsection{Tipos de datos}

La minería de datos se puede aplicar a cualquier tipo de información, lo que cambia son las técnicas que se aplican a los datos. Se diferencian entre datos relacionados y no estructurados.

\subsubsection{Bases de datos relacionales}

Estas son las que ya conozco de toda la vida para SQL, utilizadas en todo lado y según el libro,
los tipos de datos estructurados y por tablas son los que más se utilizan para aplicar técnicas
de minería de datos. La integridad de los datos se expresa a través de las restricciones de
integridad (las \textit{constraints} de toda la vida). Primero, los datos se sacan de una
consulta SQL y después pasan a un modelo de minería de datos. A aquella información sacada de
varias tablas que se requiere para hacer una tarea de minería de datos, se le conoce como
\textit{vista minable} y sale de una consulta SQL.


En las presentaciones tabulares se expresa la importancia de tener dos tipos de atributos; numéricos y categóricos.
\begin{itemize}
  \item \textbf{Numéricos}: contienen valores enteros o reales. Por ejemplo, atributos como el salario o la edad.
  \item \textbf{Categóricos o nominales}: Toman valores en un conjunto finito y preestablecido de categorías. Por ejemplo, sexo (H, M), nombre del departament, etc.
\end{itemize}


\subsubsection{Otros tipos de bases de datos}

Las relacionales son las más utilizadas hoy en día, pero existen otras, como:

\begin{itemize}
  \item \textbf{Bases de datos espaciales:} Contienen información relacionada cno el espacio físico. Cosas geográficas, imágenes médicas, redes de transporte o información de tráfico, etc. \textbf{relaciones espaciales}. La minería permite encontrar patrones entre estos datos, características de una casa en zona montañosa, planificación de nuevas líneas de metro en función de la distancia de las distintas áreas a las líneas existentes, etc. A mí se me ocurre hacer cosas con \textit{Open Street map}, pueden estar interesantes, ya que, los datos ya se tienen y puede estar todo muy chevere con eso.
  \item \textbf{Temporales:} Almacenan datos que incluyen atributos relacionados con el tiempo. La minería se utiliza para encontrar características de evolución o tendencia en el cambio de valores en la base de datos.
  \item \textbf{Documentales:} Descripciones para los objetos (documentos de texto). Incluyen documentos no estructurados, semi-estructurados o estructurados. La minería se utiliza para obtener asociaciones entre contenidos.
  \item \textbf{Multimedia:} Almacenan imágenes, audio y vídeo. Soportan gran cantidad de almacenamiento (giga bytes y todo eso). La minería se utiliza para integrar métodos de búsqueda y almacenamiento.
\end{itemize}

\subsubsection{\textit{World Wide Web}}

Este es el mayor repositorio de información que se puede tener, sin embargo, presenta muchos problemas para ser utilizada y empleada. Se categorizan en 3 campos:
\begin{itemize}
  \item \textbf{Minería del contenido:} Encontrar patrones de los datos de las páginas web.
  \item \textbf{Minería de la estructura:} Estructura se define como hipervínculos y URLs.
  \item \textbf{Minería del uso:} Navegación que hace el usuario por la web.
\end{itemize}

\subsection{Tipos de modelos}

Existen dos tipos:

\begin{itemize}
  \item \textbf{Predictivos:} Pretenden predecir o estimar valores futuros para variables de interés, denominadas \textit{variables objetivo}, usando otras variables llamadas \textit{variables independientes}. Por ejemplo, estimar la demanda de un grupo en función del gasto que se hizo en publicidad.
  \item \textbf{Descriptivo:} Identifican patrones que explican o resumen los datos, examinar las propiedades de los datos examinados, no para predecir nuevos datos. Por ejemplo, agencia de viajes que agrupa a sus usuarios para así poder enviar información a esos grupos.
\end{itemize}

\subsection{La minería de datos y el proceso de descubrimiento de conocimiento en bases de datos}

El término que se utiliza frecuentemente es el ``Análisis (inteligente) de datos'' que hace
hincapié en las técnicas de análisis estadístico. Otro muy utilizado es el ``Descubrimiento de
conocimiento en bases de datos'' (\textit{Knowledge Discovery in Databases, KDD}). Este último se
define como ``el proceso no trivial de identificar patrones válidos, novedosos, potencialmente
útiles y, en última instancia, comprensibles a partir de los datos''. Entonces, las propiedades que debe tener el conocimiento extraído deben ser:

\begin{itemize}
  \item \textbf{Valido}: Los patrones debe ser precisos para datos nuevos (se contempla un grado de incertidumbre).
  \item \textbf{Novedoso:} Aporte algo desconocido al sistema y preferiblemente para el usuario.
  \item \textbf{Útiles:} El conocimiento debe conducir a acciones que reporten algún tipo de beneficio para el usuario.
  \item \textbf{Comprensible:} Si se tiene lo contrario se puede dificultar la interpretación, revisión, validación y uso en la toma de decisiones. La información incomprensible no proporciona conocimiento.
\end{itemize}

El KDD entonces permite la obtención de modelos o patrones, también la
evaluación y posible interpretación de los mismos. El proceso de KDD sigue la
siguiente
estructura.

\begin{enumerate}
  \item Sistema de información.
  \item Preparación de los datos.
  \item Patrones.
  \item Evaluación/Interpretación/Visualización.
  \item Conocimiento.
\end{enumerate}

\newpage

\section{El proceso de extracción de conocimiento}
Para este capítulo se presentan las fases del proceso de extracción de conocimiento: Preparación de Datos, Minería de Datos, Evaluación, Difusión y Uso de Modelos.

\subsection{Las fases del proceso de extracción del conocimiento}

Es un proceso iterativo e interactivo. Iterativo porque la salida de alguna de las fases puede hacer volver a pasos anteriores, es normal tener que iterar varias veces para poder generar conocimiento de calidad. También interactivo porque el usuario o experto debe ayudar a la preparación de los datos, validar el conocimiento extraído, etc.

El KDD se organiza en 5 fases que siguen el siguiente orden:
\begin{enumerate}
  \item \textbf{Integración y recopilación de datos:} Determina las fuentes de información y donde conseguirlas. Pasan a un almacen de datos que facilite la navegación y visualización previa de los datos.
  \item \textbf{Selección, limpieza y transformación:} Se eliminan o corrigen los datos incorrectos y se decide una estrategia para los incompletos. Se proyectan los datos para considerar los que sean útiles para las tareas de minería. Las dos fases anteriores se conocen como ``preparación de datos''.
  \item \textbf{Minería de datos:} Se decide cuál es la tarea a realizar (clasificar, agrupar, etc.) y se elige el método que se va a utilizar.
  \item \textbf{Evaluación e interpretación:} Se avalúan los patrones y se analizan por los expertos, se puede volver a fases anteriores para otras iteraciones. Esto suele pasar por conflicto con el conocimiento que ya se disponía anteriormente.
  \item \textbf{Difusión:} Se usa el nuevo conocimiento y se hace participe para todos los usuarios.
\end{enumerate}

Antes de todo esto se debe establecer un objetivo para la minería de datos, esto se puede conocer como la ``ingeniería de software'' que se realiza para el cliente.

\subsection{Fase de integración y recopilación}
El \textbf{Procesamiento Transacional en Línea} (\textbf{OLTP} \textit{On-Line Transaction Processing}) cubren las necesidades diarias de una organización (facturación, control de inventario, nóminas). Pero, no sirven mucho para situaciones de toma de decisiones a largo plazo, predicción, y planificación. Lo normal es que los datos pertenezcan de diferentes lugares y fuentes. En muchos casos se pueden adquirir de bases de datos públicas, como también se pueden sacar de bases de datos privadas. Entonces, lo primero es integrar esos datos en un mismo \textbf{almacén de dato} (\textit{data warehousing})



\begin{figure}[H]
  \centering
  \tikzset{every picture/.style={line width=0.75pt}} %set default line width to 0.75pt

\begin{tikzpicture}[x=0.75pt,y=0.75pt,yscale=-1,xscale=1]
%uncomment if require: \path (0,300); %set diagram left start at 0, and has height of 300

%Flowchart: Magnetic Disk [id:dp7037311619502317]
\draw   (98,45.35) -- (98,72.65) .. controls (98,76.71) and (84.57,80) .. (68,80) .. controls (51.43,80) and (38,76.71) .. (38,72.65) -- (38,45.35)(98,45.35) .. controls (98,49.41) and (84.57,52.7) .. (68,52.7) .. controls (51.43,52.7) and (38,49.41) .. (38,45.35) .. controls (38,41.29) and (51.43,38) .. (68,38) .. controls (84.57,38) and (98,41.29) .. (98,45.35) -- cycle ;
%Flowchart: Magnetic Disk [id:dp9376614712165231]
\draw   (99,113.35) -- (99,140.65) .. controls (99,144.71) and (85.57,148) .. (69,148) .. controls (52.43,148) and (39,144.71) .. (39,140.65) -- (39,113.35)(99,113.35) .. controls (99,117.41) and (85.57,120.7) .. (69,120.7) .. controls (52.43,120.7) and (39,117.41) .. (39,113.35) .. controls (39,109.29) and (52.43,106) .. (69,106) .. controls (85.57,106) and (99,109.29) .. (99,113.35) -- cycle ;
%Flowchart: Magnetic Disk [id:dp3101985659904155]
\draw   (98,179.35) -- (98,206.65) .. controls (98,210.71) and (84.57,214) .. (68,214) .. controls (51.43,214) and (38,210.71) .. (38,206.65) -- (38,179.35)(98,179.35) .. controls (98,183.41) and (84.57,186.7) .. (68,186.7) .. controls (51.43,186.7) and (38,183.41) .. (38,179.35) .. controls (38,175.29) and (51.43,172) .. (68,172) .. controls (84.57,172) and (98,175.29) .. (98,179.35) -- cycle ;
%Right Arrow [id:dp18243938017628047]
\draw   (182,99) -- (264.26,99) -- (264.26,67) -- (319.09,131) -- (264.26,195) -- (264.26,163) -- (182,163) -- cycle ;
%Straight Lines [id:da648559349699428]
\draw    (100,59) -- (176.48,108.37) ;
\draw [shift={(179,110)}, rotate = 212.85] [fill={rgb, 255:red, 0; green, 0; blue, 0 }  ][line width=0.08]  [draw opacity=0] (8.93,-4.29) -- (0,0) -- (8.93,4.29) -- cycle    ;
%Straight Lines [id:da920556968506801]
\draw    (107,126) -- (171,127.91) ;
\draw [shift={(174,128)}, rotate = 181.71] [fill={rgb, 255:red, 0; green, 0; blue, 0 }  ][line width=0.08]  [draw opacity=0] (8.93,-4.29) -- (0,0) -- (8.93,4.29) -- cycle    ;
%Straight Lines [id:da8547662218764898]
\draw    (106,192) -- (173.58,142.77) ;
\draw [shift={(176,141)}, rotate = 143.92] [fill={rgb, 255:red, 0; green, 0; blue, 0 }  ][line width=0.08]  [draw opacity=0] (8.93,-4.29) -- (0,0) -- (8.93,4.29) -- cycle    ;
%Flowchart: Magnetic Disk [id:dp2895733072683383]
\draw   (457,102.78) -- (457,150.23) .. controls (457,157.28) and (428.35,163) .. (393,163) .. controls (357.65,163) and (329,157.28) .. (329,150.23) -- (329,102.78)(457,102.78) .. controls (457,109.83) and (428.35,115.55) .. (393,115.55) .. controls (357.65,115.55) and (329,109.83) .. (329,102.78) .. controls (329,95.72) and (357.65,90) .. (393,90) .. controls (428.35,90) and (457,95.72) .. (457,102.78) -- cycle ;

% Text Node
\draw (59,54) node [anchor=north west][inner sep=0.75pt]   [align=left] {A};
% Text Node
\draw (60,123) node [anchor=north west][inner sep=0.75pt]   [align=left] {B};
% Text Node
\draw (59,188) node [anchor=north west][inner sep=0.75pt]   [align=left] {C};
% Text Node
\draw (349.52,120.33) node [anchor=north west][inner sep=0.75pt]   [align=left] {Almacén de\\datos};
% Text Node
\draw (182,105) node [anchor=north west][inner sep=0.75pt]   [align=left] {Integración\\almacenamiento\\limpieza};
\end{tikzpicture}
\caption{Integración en un almacén de datos}
\label{fig:data-warehouse}

\end{figure}

El \textbf{Procesamiento analítico en líne (OLAP)} permite utilizar el conocimiento previo sobre el dominio de los datos para permitir su presentación a diferentes niveles de abstracción, acomodando así diferentes puntos de vista del usuario.

\subsubsection{Diferencia entre minería de datos y OLAP}

En la OLAP se combina la información que ya se tiene permitiendo realizar informes y vistas sofisticadas en tiempo real. Se utilizan para comprobar patrones y pautas hipotéticas sugeridas por el usuario con el fin de verificarlas o rechazarlas (es decir, yo tengo el almacen de datos y planteo unas preguntas que le pueda hacer; ¿$y$ actúa creciente a medida que $x$ aumenta? entonces se hace las operaciones OLAP para descartar o aceptar esas hipotésis). Es decir, se trata de un proceso exclusivamente deductivo. La minería de datos en cambio, usa los datos para encontrar patrones. Por lo tanto, es un proceso inductivo


Algo curioso que pasa con el OLAP siendo deductivo y la minería de datos siendo inductivo, es que la OLAP sí me da una conclusión a partir de una hipotésis que tenga para mis datos. Si hago una hipotésis de ¿$y$ incrementa a medida que $x$ incrementa? Entonces, se hace la respectiva consulta y puede haber una respuesta afirmando mí hipotésis o rechazandola. En cambio, al aplicar minería de datos, yo tengo la hipotésis, por ejemplo, ¿Mañana va a llover? Entonces, le puedo aplicar unos modelos de predicción, un Navier-Strokes bien exótico, esto me va a dar un modelo con cierta incertidumbre para saber si mañana puede que llueva o no. Sin embargo, puede que el modelo esté mal y mañana no llueva. Por lo tanto, en un OLAP estamos concluyendo a partir de nuestros datos, y en un proceso de minería de datos estamos sacando una probabilidad teniendo en cuenta unos datos anteriores.


Estas dos herramientas se complementan, se puede utilizar el OLAP al principio del proceso para identificar variables importantes, encontrar excepciones o interacciones.

El almacén de datos es aconsejable para la minería de datos, aunque no imprescindible.


\subsection{Fase de selección, limpieza y transformación}

``La calidad del conocimiento descubierto no sólo depende del algoritmo de minería utilizado, sino también de la calida de los datos minados.'' Después de recopilar datos el siguiente paso es seleccionar y preparar el subconjunto de datos que se va a minar, los cuales se llaman \textit{vista minable}.

Los \textit{outliers} son datos que no se ajustan al comportamiento general de los datos. Son especiales porque algunos algoritmos ignoran esos datos y otros los enfatizan, haciendolos importantes para casos en los que tener un caso excepcional o raro es algo que importa mucho. Por ejemplo, lavado de dinero, o predecir una inundación.

Los datos faltantes (\textit{missing values}) toca tener cuidado de cómo tratarlos ya que pueden ser producto de diferentes causas; el dispositivo tuvo un error en ese momento, cambios efectuados en los procedimientos o que la recopilación de datos de diferentes fuentes. Existen muchas aproximaciones que se ven durante el capítulo 4.

Una parte de la limpieza también está en seleccionar de manera correcta los atributos que van a pasar al proceso de minería de datos. Ya que, con esto se puede ahorrar tiempo de cálculo en la máquina y también se ahorran recursos en general.

Los tipos de datos pueden modificarse para facilitar el uso de algunas técnicas. Por ejemplo, se tiene que una vivienda puede tener un tipo de tamaño (pequeño, mediano, y grande). Se puede modificar para que pasen a ser numéricos y por intervalos. Es decir, que el pequeño sea de 0 a 3, mediando de 4 a 7, y grande de 8 a 10. Este proceso también se puede hacer de manera inversa para tener las categorías de los tamaños para las casas.


\end{document}
